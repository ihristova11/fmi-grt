\documentclass[a4paper,12pt]{article}
\usepackage{amsmath}
\usepackage{graphicx}
\usepackage{hyperref}
\usepackage[utf8]{inputenc}
\usepackage[english,bulgarian]{babel}
\usepackage{amsthm}
\usepackage{mathtools}
\usepackage{graphicx}
\usepackage{amsfonts}
\usepackage{amssymb}

\title{Решения на задачите по ОТО}
\author{Ирина Христова, 82071}
\begin{document}
\maketitle
\pagebreak
\section*{Задача 1.}
Спрямо СТО времето в различните инерциални отправни системи, движещи се с 
различни скорости спрямо една абсолютна (или "лабораторна") инерциална отправна система,
тече по различен начин.

Оттук възниква и т.нар. "парадокс на близнаците". Нека приемем, че имаме двама близнаци, 
единият от които е в покой спрямо инерциалната (приемаме я за такава) отправна система, свързана
със Земята, а другият близнак е в покой спрямо интерциалната отправна система, свързана 
с космичеки кораб, който се движи с постоянна скорост около Земята.

Нека разгледаме близнака, намиращ се на Земята. От негова гледна точка
космическият кораб се отдалечава с постоянна скорост и спрямо СТО времето в отправната система, 
свързана с кораба, тече по-бавно. Това означава, че след като пътуващият на космическия кораб близнак се завърне на Земята, 
той трябва е по-млад от близнака, който през цялото време е бил на Земята.

Нека разгледаме близнака, движеш се на космическия кораб. От негова гледна точка Земята се отдалечава от него с постоянна 
скорост, т.е. времето тече по-бавно за близнака, който се намира на Земята. Това означава, че след като двамата близнаци се срещнат отново, 
от гледна точка на близнака в космическия кораб, другият близнак ще бъде по-млад от него.

Така стигаме до извода, че спрямо който и да е от двамата близнаци, другият ще е по-млад след като се срещнат, което не е възможно.

До това противоречие стигнахме след предположението, че двете отправни системи са напълно равноправни, 
защото СТО ни казва, че физическите явления във всеки две инерциални отправни системи са напълно еднакви.

Решението на този "парадокс" лежи в това, че за да се върне космическият кораб на Земята, трябва да промени посоката на скоростта си. 
Тази промяна на посоката на скоростта неизбежно е свързана с появата на ускорение, защото дори и корабът да се движи с постоянна по големина скорост през цялото време, 
неизбежно ще възникне поне центростремително ускорение. Това означава, че близнакът, който е на космическия кораб, ще може да разбере с физически експеримент на кораба, че изпитва ускорение
и неговата отправна система не е инерциална. Така възниква неравнопоставеност между 
двете отправни системи и наистина двамата близнаци ще са на различни години след като се срещнат (близнакът на Земята ще бъде по-стар).

Така се потвърждава и от математическия апарат на СТО.

\section*{Задача 5.}
Метрика на Шварцшилд:
\begin{equation*}
    ds^2 = \left(1 - \frac{\alpha}{r}\right)c^2dt^2 - \frac{dr^2}{1 - \frac{\alpha}{r}} - r^2\left[ d\theta^2 + \sin^2\theta d\varphi^2 \right]
\end{equation*}
В координати $(ct, r, \theta, \varphi)$ само диагоналните
компоненти на метричния тензор са ненулеви: 
\begin{equation*}
\begin{pmatrix}
    1-\frac{\alpha}{r} & 0 & 0 & 0\\
    0 & -\frac{1}{1-\frac{\alpha}{r}} & 0 & 0 \\
    0 & 0 & -r^2 & 0 \\
    0 & 0 & 0 & -r^2\sin^2\theta
\end{pmatrix}
\end{equation*}

Ще пресметнем символите на Кристофел по формулата:
\begin{equation*}
    \varGamma^m_{ij} = \frac{1}{2}g^{mk}(g_{ki,j} + g_{kj,i}-g_{ij,k})
\end{equation*}

Тъй като недиагоналните членове са нула, т.е. $g^{mk}=0$ при $m \neq k$, следва:
\begin{equation*}
    \varGamma^m_{ij} = \frac{1}{2}g^{mm}(g_{mi,j}+g_{mj,i} - g_{ij, m})
\end{equation*}

Тъй като компонентите на метричния тензор не зависят от $t$ или $\varphi$, то:

    $g_{ij, 0} = \frac{\partial g_{ij}}{\partial t} =0$ и 
    $g_{ij, 3} = \frac{\partial g_{ij}}{\partial \varphi} =0$

    Пресмятаме: 

    \begin{equation}
        \begin{aligned}
        \varGamma^0_{01} = \frac{1}{2}g^{00}\left(\frac{\partial g_{00}}{\partial r} + \frac{\partial g_{01}}{\partial t} - \frac{\partial g_{01}}{\partial t}\right) =\\
        = \frac{1}{2}\left( 1- \frac{\alpha}{r}\right)^{-1} \frac{\partial}{\partial r}\left(1 - \frac{\alpha}{r}\right) =\\
        = \frac{1}{2}.\frac{1}{1 - \frac{\alpha}{r}}.\frac{\alpha}{r^2} = \frac{\alpha}{2r^2}.\frac{1}{\frac{r-\alpha}{r}} =\\ 
        = \frac{\alpha}{2r(r-\alpha)}
        \end{aligned}
    \end{equation}
    \newline

    \begin{equation}
        \begin{aligned}
        \varGamma^1_{11} = \frac{1}{2}g^{11}\left(\frac{\partial g_{11}}{\partial r} + \frac{\partial g_{11}}{\partial r} - \frac{\partial g_{11}}{\partial r}\right) =\\
        = \frac{1}{2}\left[ - \left( 1- \frac{\alpha}{r}\right)\right] \frac{\partial}{\partial r}\left(- \frac{1}{1-\frac{\alpha}{r}}\right) =\\
        = \frac{1}{2}.\left( 1- \frac{\alpha}{r}\right).\frac{r -\alpha - r}{(r-\alpha)^2} =\\ 
        = \frac{1}{2}.\frac{r-\alpha}{r}.\frac{(-\alpha)}{(r-\alpha)^2} = \\
        = -\frac{\alpha}{2r(r-\alpha)}
        \end{aligned}
    \end{equation}
    \newline

    \begin{equation}
        \begin{aligned}
        \varGamma^1_{00} = \frac{1}{2}g^{11}\left(\frac{\partial g_{10}}{\partial t} + \frac{\partial g_{10}}{\partial t} - \frac{\partial g_{00}}{\partial r}\right) =\\
        = -\frac{1}{2}\left( 1- \frac{\alpha}{r}\right).(-1).\frac{\partial}{\partial r}.\left( 1- \frac{\alpha}{r}\right) =\\
        = \frac{1}{2}.\left( 1- \frac{\alpha}{r}\right).\frac{\alpha}{r^2} =\\ 
        = \frac{1}{2}.\frac{r-\alpha}{r}.\frac{(-\alpha)}{(r-\alpha)^2} = \\
        = -\frac{\alpha(r-\alpha)}{2r^3}
        \end{aligned}
    \end{equation}
    \newline

    \begin{equation}
        \begin{aligned}
        \varGamma^2_{21} = \frac{1}{2}g^{22}\left(\frac{\partial g_{22}}{\partial t} + \frac{\partial g_{21}}{\partial \theta} - \frac{\partial g_{21}}{\partial \theta}\right) =\\
        = \frac{1}{2}\left(- \frac{1}{r^2}\right).\left[\frac{\partial}{\partial r}.(-r^2)\right] = - \frac{1}{2r^2}(-2r) = \frac{1}{r}
        \end{aligned}
    \end{equation}
    \newline

    \begin{equation}
        \begin{aligned}
        \varGamma^3_{31} = \frac{1}{2}g^{33}\left(\frac{\partial g_{33}}{\partial r} + \frac{\partial g_{31}}{\partial \varphi} - \frac{\partial g_{31}}{\partial \varphi}\right) =\\
        = -\frac{1}{2}\frac{1}{r^2\sin^2\theta}.\frac{\partial}{\partial r}(-r^2\sin^2\theta) = \\
        = \frac{1}{2} \frac{1}{r^2\sin^2\theta}.2r\sin^2\theta = \frac{1}{r}
        \end{aligned}
    \end{equation}
    \newline

    \begin{equation}
        \begin{aligned}
        \varGamma^1_{22} = \frac{1}{2}g^{11}\left(\frac{\partial g_{12}}{\partial \theta} + \frac{\partial g_{12}}{\partial \theta} - \frac{\partial g_{22}}{\partial r}\right) =\\
        = -\frac{1}{2}\left(1- \frac{\alpha}{r}\right).(-1).\frac{\partial}{\partial r}.(-r^2) = \\
        = -\frac{1}{2} \left(1-\frac{\alpha}{r}\right).2r = \\
        = - \frac{r-\alpha}{r}.r = \alpha - r
        \end{aligned}
    \end{equation}
    \newline

    \begin{equation}
        \begin{aligned}
        \varGamma^1_{33} = \frac{1}{2}g^{11}\left(\frac{\partial g_{13}}{\partial \varphi} + \frac{\partial g_{13}}{\partial \varphi} - \frac{\partial g_{33}}{\partial r}\right) =\\
        = -\frac{1}{2}\left(1- \frac{\alpha}{r}\right).(-1).\frac{\partial}{\partial r}.(-r^2\sin^2\theta) = \\
        = -\frac{1}{2} \left(1-\frac{\alpha}{r}\right).2r\sin^2\theta = \\
        = - (\alpha - r)\sin^2\theta
        \end{aligned}
    \end{equation}
    \newline

    \begin{equation}
        \begin{aligned}
        \varGamma^2_{33} = \frac{1}{2}g^{22}\left(\frac{\partial g_{23}}{\partial \varphi} + \frac{\partial g_{23}}{\partial \varphi} - \frac{\partial g_{33}}{\partial \theta}\right) =\\
        = \frac{1}{2}\left(- \frac{1}{r^2}\right).(-1).\frac{\partial}{\partial \theta}.(-r^2\sin^2\theta) = \\
        = -\frac{1}{2r^2}.2r^2\sin\theta\cos\theta = \\
        = -\sin\theta\cos\theta
        \end{aligned}
    \end{equation}
    \newline

    \begin{equation}
        \begin{aligned}
        \varGamma^3_{32} = \frac{1}{2}g^{33}\left(\frac{\partial g_{33}}{\partial \theta} + \frac{\partial g_{32}}{\partial \varphi} - \frac{\partial g_{32}}{\partial \varphi}\right) =\\
        = -\frac{1}{2}\frac{1}{r^2\sin^2\theta}.\frac{\partial}{\partial \theta}.(-r^2\sin^2\theta) = \\
        = \frac{1}{2r^2\sin^2\theta}.r^2\sin\theta\cos\theta = \\
        = \cot\theta
        \end{aligned}
    \end{equation}
    \newline
    Окончателно получаваме за ненулевите символи на Кристофел:


    \begin{equation*}
        \begin{aligned}
        \varGamma^0_{01} = \varGamma^1_{11} = \frac{\alpha}{2r(r-\alpha)} \\ 
        \varGamma^1_{00} = \frac{\alpha(r-\alpha)}{2r^3} \\ 
        \varGamma^1_{22} = \alpha - r \\
        \varGamma^2_{21} = \varGamma^3_{31}\frac{1}{r} \\
        \varGamma^1_{33} = (\alpha - r)\sin^2\theta \\
        \varGamma^2_{33} = - \sin\theta\cos\theta \\
        \varGamma^3_{32} = \cot\theta
    \end{aligned}
    \end{equation*}

    Тензор на Ричи:
    \begin{equation*}
        R_{ik} = R^a_{iak} = \partial_k\varGamma^a_{ia} - \partial_a\varGamma^a_{ik}-\varGamma^a_{ba}\varGamma^b_{ik} + \varGamma^a_{bk}\varGamma^b_{ai}
    \end{equation*}

    Тъй като сме в геометрия на Шварцшилд, имаме пълна симетрия и ако си представим
    сферично-симетрична черна дупка в геометрия на Шварцшилд, ще имаме 
    цялата маса, съсредоточена в сингулярност, а извън нея всички компоненти на тензора на 
    Ричи ще бъдат нула.

    $R_{ik}=0 \forall i, k \in [0, 3]$

    Можем нагледно да го покажем за първата компонента:

    \begin{equation*}
        \begin{aligned}
            R_{00}=R^a_{0a0} = \partial_0\varGamma^a_{0a} - \partial_a\varGamma^a_{00}-\varGamma^a_{ba}\varGamma^b_{00}+\varGamma^a_{bo}\varGamma^b_{a0}=\\
            = - \partial_a\varGamma^a_{00} - \varGamma^a_{1a}\varGamma^1_{00}+\varGamma^0_{bo}\varGamma^b_{00} =\\
            = -\partial_a\varGamma^1_{00}-\varGamma^1_{11}\varGamma^1_{00}-\varGamma^2_{12}\varGamma^1_{00}-\varGamma^3_{13}\varGamma^1_{00}+\varGamma^0_{10}\varGamma^1_{00} = \\
            =- \frac{\partial}{\partial r}\left[ \frac{\alpha(r-\alpha)}{2r^3} \right] + \frac{\alpha}{2r(r-\alpha)}.\frac{\alpha(r-\alpha)}{2r^3}-2.\frac{1}{r}\frac{\alpha(r-\alpha)}{2r^3}
            + \frac{\alpha}{2r(r-\alpha)}.\frac{\alpha(r-\alpha)}{2r^3} = \\
            = -\alpha.\frac{2r^3 - 6r^3+6\alpha r^3}{4r^6} + \frac{\alpha^2}{2r^4} - \frac{\alpha(r-\alpha)}{r^4} = \\
            = \frac{4\alpha r^3 - 6 \alpha^2 r^2 + 2r^2\alpha^2-4r^3\alpha+4r^2\alpha^2}{4r^6} = 0
        \end{aligned}
    \end{equation*}

    Аналогично и за останалите компоненти $R_{ik}=0$.\\
    Скаларната кривина е следната $R=g^{ik}R_{ik}=0$.

    Тензорът на Риман: 
    \begin{equation*}
        R^i_{jkl}=\frac{\partial\varGamma^0_{10}}{\partial x^l} - \frac{\partial \varGamma^i_{jl}}{\partial x^k} + \varGamma^i_{al}\varGamma^a_{jk}-\varGamma^i_{ak}\varGamma^a_{jl}
    \end{equation*}

    Ненулевите компоненти са следните: 
    \begin{equation*}
        \begin{aligned}
            R^0_{101} = \frac{\partial \varGamma^0_{10}}{\partial x^1} - \frac{\partial\varGamma^0_{11}}{\partial x^0}+ \varGamma^0_{a1}\varGamma^a_{10}
            - \varGamma^0_{a0}\varGamma^a_{11}= \\
            = \frac{\partial}{\partial r} \left( \frac{\alpha}{2r(r-\alpha)} \right)
            + \varGamma^0_{01}\varGamma^0_{10} - \varGamma^0_{10}\varGamma^1_{11} =\\
            = \frac{\alpha}{2}\frac{\partial}{\partial r}\left( \frac{1}{r^2-\alpha r} \right) + \left( \frac{\alpha}{2r(r-\alpha)} \right)^2 + \left( \frac{\alpha}{2r(r-\alpha)} \right)= \\
            = - \frac{(2r-\alpha)\alpha}{2(r^2-\alpha r)^2} + 2\frac{\alpha^2}{4r^2(r-\alpha)^2} = \\
            = \frac{\alpha(\alpha - 2r)}{2r^2(r-\alpha)^2} + \frac{2\alpha^2}{4r^2(r-\alpha)^2} = \frac{\alpha^2-2\alpha r + \alpha^2}{2r^2(r-\alpha)^2}= \\
            = \frac{2\alpha^2-2\alpha r}{2r^2(r-\alpha)} = -\frac{2\alpha(\alpha-r)}{2r^2(\alpha-r)^2} = - \frac{\alpha}{r^2(\alpha - r)}
        \end{aligned}
    \end{equation*}

    Намираме 
    \begin{equation*}
        \begin{aligned}
            R_{0101} = g_{m0}R^0_{101} = g_{00}R^0_{101} = -\left( 1-\frac{\alpha}{r}\right).\frac{\alpha}{r^2(\alpha -r)} = - \frac{r-\alpha}{r}.\frac{\alpha}{r^2(\alpha-r)} = \frac{\alpha}{r^3}
        \end{aligned}
    \end{equation*}
    \newline
    \begin{equation*}
        \begin{aligned}
            R^0_{202} = \frac{\partial\varGamma^0_{20}}{\partial x^2} - \frac{\partial\varGamma^0_{22}}{\partial x^0} + \varGamma^0_{a2}\varGamma^a_{20}-\varGamma^0_{a0}\varGamma^a_{22}
            = -\varGamma^0_{10}\varGamma^1_{22} = -\frac{\alpha}{2r(r-\alpha)}.(\alpha-r)=\frac{\alpha}{2r}
            \\
            R_{0202}=g_{00}R^0_{202} = \frac{\alpha}{2r}\left( 1-\frac{\alpha}{r} \right)
        \end{aligned}
    \end{equation*}
    \newline
    \begin{equation*}
        \begin{aligned}
            R^0_{303} = \frac{\partial \varGamma^0_{30}}{\partial x^3} - \frac{\partial\varGamma^0_{33}}{\partial x^0} + \varGamma^0_{a3}\varGamma^a_{30} - \varGamma^0_{a0}\varGamma^a_{33}
            = - \varGamma^0_ {10}\varGamma^1_{33} = -\frac{\alpha}{2r(r-\alpha)}.(\alpha-r)\sin^2\theta = \frac{\alpha\sin^2\theta}{2r}
            \\
            R_{0303}=g_{00}R^0_{303} = \left( 1- \frac{\alpha}{r} \right) \frac{\alpha\sin^2\theta}{2r} = \frac{\alpha(r-\alpha)\sin^2\theta}{2r^2}
        \end{aligned}
    \end{equation*}
    \newline
    \begin{equation*}
        \begin{aligned}
            R^1_{212} = \frac{\partial\varGamma^1_{21}}{\partial x^2} - \frac{\partial\varGamma^1_{22}}{\partial x^1} + \varGamma^1_{a2} + \varGamma^a_{21} - \varGamma^1_{a1}\varGamma^a_{22}
            = - \frac{\partial}{\partial r} (\alpha-r) + \varGamma^1_{12}\varGamma^1_{21}+\varGamma^1_{22}\varGamma^2_{21}-\varGamma^1_{11}\varGamma^1_{22} = \\
            =1 + (\alpha - r)\frac{1}{r}+\frac{\alpha}{2r(r-\alpha)}.(\alpha-r) = 1+\frac{\alpha-r}{r} - \frac{\alpha}{2r(r-\alpha)}(\alpha-r)=1+\frac{\alpha-r}{r}-\frac{\alpha}{2r} = \\
            = \frac{2r+2\alpha-2r-\alpha}{2r} = \frac{\alpha}{2r}
            \\
            R_{1212}=g_{11}R^1_{212}=-\frac{1}{1-\frac{\alpha}{r}}.\frac{\alpha}{2r}= - \frac{r}{r-\alpha}.\frac{\alpha}{2r}=-\frac{\alpha}{2(r-\alpha)} =\frac{\alpha}{2(\alpha-r)}
        \end{aligned}
    \end{equation*}
    \newline
    \begin{equation*}
        \begin{aligned}
            R^1_{313}=\frac{\partial\varGamma^1_{31}}{\partial x^3} - \frac{\partial\varGamma^1_{33}}{\partial x^1}+\varGamma^1_{a3}\varGamma^a_{31}-\varGamma^1_{a1}\varGamma^a_{33}
            = -\frac{\partial}{\partial r} [(\alpha-r)\sin^2\theta] + \varGamma^1_{33}\varGamma^3_{31} - \varGamma^1_{11}\varGamma^1_{33} = \\
            = \sin^2\theta + (\alpha-r)\sin^2\theta\frac{1}{r} + \frac{\alpha}{2r(r-\alpha)}(\alpha-r)\sin^2\theta
            = \sin^2\theta\left[ 1 + \frac{\alpha -r}{r} - \frac{\alpha}{2r} \right] = \sin^2\theta.\frac{\alpha}{2r}
            \\
            R_{1313}=g_{11}R^1_{313} = -\frac{1}{1-\frac{\alpha}{r}}.\frac{\alpha}{2r}.\sin^2\theta = \frac{\alpha\sin^2\theta}{2(\alpha-r)}
        \end{aligned}
    \end{equation*}
    \newline

    \begin{equation*}
        \begin{aligned}
            R^2_{323} = \frac{\partial\varGamma^2_{32}}{\partial x^3} - \frac{\partial\varGamma^2_{33}}{\partial x^2} + \varGamma^2_{a3}\varGamma^2_{32} - \varGamma^2_{a2}\varGamma^2_{33} = -\frac{\partial}{\partial\theta}(-\sin\theta\cos\theta) + \varGamma^2_{33}\varGamma^3_{32}-\varGamma^2_{22}\varGamma^2_{33}-\varGamma^2_{12}\varGamma^1_{33}= \\
            = \cos^2\theta-\sin^2\theta + (-\sin\theta\cos\theta)\cot\theta - \frac{1}{r}(\alpha-r)\sin^2\theta = -\sin^2\theta + \frac{r-\alpha-r}{r}=-\frac{\alpha}{r}\sin^2\theta
        \end{aligned}
    \end{equation*}

    \begin{equation*}
        R_{2323}=g_{22}R^2_{323}=-r^2.\left( -\frac{\alpha}{r} \right)\sin^2\theta=\alpha r\sin^2\theta
    \end{equation*}

    Окончателно: 
    \begin{equation*}
        R_{0101} = \frac{\alpha}{r^3} 
    \end{equation*}
    \begin{equation*}
        R_{0202} = \frac{\alpha(r-\alpha)}{2r^2} 
    \end{equation*}
    \begin{equation*}
        R_{0303} = \frac{\alpha(r-\alpha)\sin^2\theta}{2r^2} 
    \end{equation*}
    \begin{equation*}
        R_{1212} = \frac{\alpha}{2(\alpha-r)} 
    \end{equation*}
    \begin{equation*}
        R_{1313} = \frac{\alpha\sin^2\theta}{2(\alpha-r)} 
    \end{equation*}
    \begin{equation*}
        R_{2323} = \alpha r\sin^2\theta
    \end{equation*}

    \section*{Задача 6.}
    ФН: 82071 $\rightarrow$ Меркурий

    Точна формула за отместване на перихелия: \\
    Кеплеровото движение на една планета около Слънцето (т.е. в неговото гравитационно поле на Шварцшилд), може да се даде със следните 
    формули:

    $
        r(\varphi) = \frac{p}{1-e.\cos_q u}
    $
    или 
    $
        \frac{1}{r} = \frac{cn^2(\frac{u}{2})}{r_{min}} + \frac{sn^2(\frac{u}{2})}{r_{max}}
    $, 
    където $r(\varphi)$ е елиптична функция, а $r_{min}$ и $r_{max}$ са най-малкото и най-голямото
    разстояние от Слънцето до планетата. \\ Също така имаме, че: \\
    $
        r_{min} = (1-e)r_{\text{средно}}
    $ и 
    $
        r_{max}=(1+e)r_{\text{средно}}
    $, където $e$ е ексцентрицитетът на орбитата.
    \\ Също имаме, че: \\
    $p = \frac{2}{\frac{1}{r_{max}} + \frac{1}{r_{min}}}$.

    Използваме: 
    \begin{equation*}
        \left( \frac{dr}{d\varphi} \right)^2 = \frac{c^2_1-h}{c^2_2}r^4 + \frac{ah}{c^2_2}r^3 - r^2+ar = \frac{dr(r-r_1)(r-r_{min})(r_{max}-r)}{r_1r_{min}r_{max}}
    \end{equation*}
    и от формулите на Виет: $\frac{1}{\alpha} = \frac{1}{r_1} + \frac{1}{r_{min}} + \frac{1}{r_{max}}$ \\ 
    $\frac{1}{r_1} = \frac{1}{\alpha} - \frac{1}{r_{min}} - \frac{1}{r_{max}}$.
    \\
    Трябва да пресметнем елиптичните интеграли от първи род: 
    \begin{equation*}
        \begin{aligned}
            K = \int_{0}^{1} \frac{ds}{\sqrt[]{(1-s^2)(1-\kappa^2s^2)}}, \kappa^2 = \frac{\frac{1}{r_{min}} - \frac{1}{r_{max}}}{\frac{1}{r_1} - \frac{1}{r_{max}}} \\
            K' = \int_{0}^{1} \frac{ds}{\sqrt[]{(1-s^2)(1-\kappa^{'2}s^2)}}, \kappa^{'2} = 1-\kappa^2 \\
    \end{aligned}
    \end{equation*}

    Реалният период на $r$ по $\varphi$ е равен на:

    \begin{equation*}
        T = \frac{4K}{\sqrt[]{\frac{\alpha}{r_1} - \frac{\alpha}{r_{max}}}} = \frac{2\pi}{M\left(\sqrt[]{\frac{\alpha}{r_1} - \frac{\alpha}{r_{max}}}, \sqrt[]{\frac{\alpha}{r_1} - \frac{\alpha}{r_{min}}}\right)} - 2\pi
    \end{equation*}

    За Слънцето: 
    \begin{equation*}
        \alpha = 2\frac{GM}{c^2} \approx 2. \frac{6,679.10^{-11}m^3kg^{-1}s^{-2} . 1,989.10^{30}kg}{(299 792 458m/s)^2} \approx 2, 95km
    \end{equation*}

    \begin{equation*}
        q- \frac{\alpha}{r_{min}} - \frac{2\alpha}{r_{max}} \approx 0, 99999985
    \end{equation*}

    \begin{equation*}
        \begin{aligned}
            1 - \frac{2\alpha}{r_{min}} - \frac{\alpha}{r_{max}} \approx 0,99999983 \\
            \implies \varDelta\varphi \approx 2\pi \left( \frac{1}{0,99999984} - 1 \right) \approx 1, 005.10^{-6}
        \end{aligned}
    \end{equation*}

    Отместването по перихелия по точната формула е: \\ 
    $\approx \varDelta\varphi \frac{100\text{ години}}{87, 969 \text{ дни}} \approx 41''$

    Приближение: 
    \begin{equation*}
        \begin{aligned}
            \varDelta\varphi \frac{2\pi}{M\left( 1-\frac{\alpha}{2r_{min}} - \frac{\alpha}{r_{max}}, 1-\frac{\alpha}{r_{min}} - \frac{\alpha}{2r_{max}} \right)} - 2\pi \approx 
            \frac{2\pi}{1 - \frac{3\alpha}{\varphi r_{min}} - \frac{3\alpha}{4r_{max}}} - 2\pi \approx \\
            \approx \frac{3}{2}.\frac{\pi\alpha}{\frac{1}{r_{min}} + \frac{1}{r_{max}}} = \frac{3}{2}\pi\alpha \frac{1+e+1-e}{r_{\text{средно}(1-e)(1+e)}} = \frac{3\pi\alpha}{r_{\text{средно}(1-e^2)}} \rightarrow \text{формула на Айнщайн}
        \end{aligned}
    \end{equation*}

    \begin{equation*}
        \varDelta\varphi \approx \frac{3\pi\alpha}{r_{text{средно}}(1-e^2)} \approx 43''
    \end{equation*}

    Получихме резултати за отместването на перихелия на Меркурий по точната формула и по тази на Айнщайн.
\end{document}